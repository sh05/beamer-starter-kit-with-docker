\section{Beamerとは}
\subsection{紹介}
\begin{frame}
    \frametitle{\insertsection}
    \beameris
    \bigskip
    \\つまり
    \assert{プレゼン資料をテキスト(\LaTeX)で管理できる}
\end{frame}

\begin{frame}
    \frametitle{何が嬉しい}
    \begin{itemize}
        \item 好きなエディタで作成できる
        \item 動的に処理してくれる
        \begin{itemize}
            \item ページ数の分母
            \item 目次
            \begin{itemize}
                \item 現在の各section,sebsectionをハイライト
            \end{itemize}
            \item 各section,sebsectionの開始時に自動でスライドを挿入
            \item 外部だけでなく資料内部へもリンクを作成
        \end{itemize}
        \item 構造的に資料作成できる(主観)
    \end{itemize}
\end{frame}

\begin{frame}
  \frametitle{目次がコマンドで自動生成される}
  \tableofcontents
  \bigskip
  \footnotesize
  \begin{itemize}
      \item \textbackslash tableofcontentsで自動生成される
      \item 各項目をPDFのビューアでクリックすると移動できる
  \end{itemize}
\end{frame}

\begin{frame}
  \frametitle{段組もできる}
    \begin{multicols}{2}
      \tableofcontents[sections={1}]
      \bigskip
      \tableofcontents[sections={2}]
      \columnbreak
      \tableofcontents[sections={3}]
      \bigskip
      \tableofcontents[sections={4}]
    \end{multicols}
    \bigskip
    \footnotesize
    \begin{itemize}
        \item \textbackslash tableofcontentsで自動生成される
        \item 各項目をPDFのビューアでクリックすると移動できる
    \end{itemize}
\end{frame}

\begin{frame}
  \frametitle{ \LaTeX にできることはできる}
  もちろん数式をきれいに書ける 
  \begin{displaymath}
      \int^{b}_{a} f(x) dx = \lim_{n \to \infty} \sum^{n-1}_{i=0} f(x_{i}) \Delta x
  \end{displaymath}
  コマンドに登録して再利用できる\\
    例えば\textbackslash beamerisに\small{\textcolor{blue!35}{\beameris}}を登録できる。また、後述のTikzと相性が良い
\end{frame}

\subsection{基本}
% コード埋め込み用
\begin{lrbox}{\codebox}
    \begin{lstlisting}[language=TeX]
    \begin{frame}
    ...
    \end{frame}
    \end{lstlisting}
\end{lrbox}

\begin{frame}[fragile]
  \frametitle{基本}
  下記のframeがスライド一枚に該当\\
  \medskip
  \usebox{\codebox}
  \bigskip
  文法などについてはとても親切なページが多いので詳しくはそちらを参照 \cite{BeamerT10:online} \cite{Beamer読本73:online} (クリックで参考文献)、ここでは主機能の紹介に留める
\end{frame}


\begin{frame}{}
  \frametitle{アニメーション}
  \begin{itemize}
    \item 1枚目
      \onslide<2->
    \item 2枚目
      \onslide<3->
    \item 3枚目
      \begin{itemize}
        \item 右下のページ数は変わらない
        \item handout版ではまとまってる
      \end{itemize}
  \end{itemize}
\end{frame}

\begin{frame}{}
  \frametitle{表}
    \begin{multicols}{2}

  例:表\ref{tab:month} は和風月名と英語での月表現のリストだが、試験によく出るので要注意だ
      \columnbreak
  \small
  \begin{table}[htb]
      \begin{center}
          \caption{和風月名と月の英語表現}
          \label{tab:month}
          \begin{tabular}{rcc}
              \hline
              月 & 和風月名 & 英語 \\
              \hline \hline
              1 & 睦月 &   January \\
              2 & 如月 &   February \\
              3 & 弥生 &   March \\
              4 & 卯月 &   April \\
              5 & 皐月 &   May	 \\
              6 & 水無月 & June \\
              7 & 文月 &   July \\
              8 & 葉月 &   August	 \\
              9 & 長月 &   September \\
              10 & 神無月 & October	 \\
              11 & 霜月 &  November  \\
              12 & 師走 &  December  \\
              \hline
          \end{tabular}
      \end{center}
  \end{table}
    \end{multicols}
\end{frame}

\begin{frame}{}
  \frametitle{表とアニメーション}
    \begin{multicols}{2}

  例:表\ref{tab:monthanime} は和風月名と英語での月表現のリストだが、試験によく出るので要注意だ
      \columnbreak
  \small
  \begin{table}[htb]
      \begin{center}
          \caption{和風月名と月の英語表現}
          \label{tab:monthanime}
          \begin{tabular}{rcc}
              \hline
              月 & 和風月名 & 英語 \\
              \hline \hline
              1 & 睦月 &   January \\
              2 & 如月 &   February \\
              3 & 弥生 &   March \onslide<2-3> \\
              4 & 卯月 &   April \\
              5 & 皐月 &   May	 \\
              6 & 水無月 & June \\
              7 & 文月 &   July \\
              8 & 葉月 &   August	 \\
              9 & 長月 &   September \\
              10 & 神無月 & October	 \onslide<3-4>  \\
              11 & 霜月 &  November  \\
              12 & 師走 &  December  \\
              \hline
          \end{tabular}
      \end{center}
  \end{table}
    \end{multicols}
\end{frame}

\begin{frame}
  \frametitle{図}
  図 \ref{fig:hlm}は髪を長く伸ばしている、ロングヘアー(ロングヘア・ロン毛)の男性のイラスト。いらすとや\cite{かわいいフリー素30:online}のランダムで出た画像で特に意味はない。
  \begin{figure}[]
    \centering
      \includegraphics[width=0.3\linewidth]{hair_long_man.png}
      \caption{ロングヘアーの男性のイラスト}
      \label{fig:hlm}
    \end{figure}
\end{frame}
