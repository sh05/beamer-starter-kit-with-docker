\section{入門}
\subsection{条件確認と準備}
\begin{frame}{}
  \frametitle{前提条件}
  \begin{itemize}
    \item git
    \item Docker-Compose
  \end{itemize}
\end{frame}

\begin{frame}[fragile]
  \frametitle{準備}
  gitがインストールされている場合
  \scriptsize
  \begin{lstlisting}[language=sh]
  git clone \
  https://github.com/sh05/beamer-starter-kit-with-docker.git
  \end{lstlisting}
  \normalsize
  \medskip
  gitがインストールされていない場合
  \begin{enumerate}
      \item \footnotesize{\textit{https://github.com/sh05/beamer-starter-kit-with-docker}}にアクセス
      \item \textit{Code}をクリック
      \item \textit{Download ZIP}を選択
      \item 解凍して端末で表示
  \end{enumerate}
\end{frame}

\subsection{とりあえず動かしてみる}
\begin{frame}[fragile]
  \frametitle{\insertsubsection}
  ※初回は大きなデータのダウンロードが始まります
  \scriptsize
  \begin{lstlisting}[caption=雑な実行, language=sh]
  docker-compose up
  \end{lstlisting}
  \begin{lstlisting}[caption=丁寧な実行, language=sh]
  # 初回
  docker pull paperist/alpine-texlive-ja:latest
  docker-compose build
  docker-compose up
  # 以降
  docker-compose up
  \end{lstlisting}
  \normalsize
    PDFファイルが\highlight{\textit{texsrc/main.pdf}}に書き出される
\end{frame}

