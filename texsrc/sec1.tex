\section{概要}

\subsection{サブセクション3}

\begin{frame}{}
  \frametitle{エディタでスライドを作成できる}

    下記のframeがスライド一枚に該当

    \begin{center}
      \textbackslash begin\{frame\} \\
      ... \\
      \textbackslash end\{frame\}
    \end{center}

    \bigskip

    基本は\TeX 記法なのでもちろん数式をきれいに書ける

      \begin{displaymath}
        \int^{b}_{a} f(x) dx = \lim_{n \to \infty} \sum^{n-1}_{i=0} f(x_{i}) \Delta x
      \end{displaymath}
\end{frame}

\begin{frame}{}
  \frametitle{アニメーションもできる}
  \begin{itemize}
    \item 1枚目
      \onslide<2->
    \item 2枚目
      \begin{itemize}
        \item 右下のページ数は変わらない
        \item handout版ではまとまってる
      \end{itemize}
  \end{itemize}
\end{frame}

\begin{frame}{}
  \frametitle{定義すると再利用できる}
  下記なども呼び出せる
  \begin{description}
    \item [セクション] \insertsection
    \item [サブセクション] \insertsubsection
  \end{description}
\end{frame}
