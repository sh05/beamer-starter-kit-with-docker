\section{セクションA}

\subsection{サブセクション1}
\begin{frame}{}
  \frametitle{アニメーションもできる}
  \begin{itemize}
    \item 1枚目
      \onslide<2->
    \item 2枚目
    \item 右下のページ数は変わらない
    \item handout版ではまとまってる
  \end{itemize}
\end{frame}

\subsection{サブセクション2}
\begin{frame}{}
  \frametitle{\insertsubsection}
  下記なども呼び出せる
  \begin{description}
    \item [セクション] \insertsection
    \item [サブセクション] \insertsubsection
  \end{description}
\end{frame}

\subsection{サブセクション3}
\begin{frame}{}
  \frametitle{\insertsubsection}
  もちろんtex記法できれいな数式を書ける
  $$\displaystyle e^{i\varphi} = \cos{\varphi} + i\sin{\varphi} $$
\end{frame}

\subsection{サブセクション4}
\begin{frame}{}
  \frametitle{\insertsubsection}
  git pull
  docker-compose up
\end{frame}
