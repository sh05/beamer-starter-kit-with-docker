\documentclass[dvipdfmx, handout]{beamer}

% \include{hogehoge} でhogehoge.texを読み込む

% プリアンブル読み込み
% 設定のようなもの
% タイトル
\title{Beamer Starter Kit with Docker}
\author{https://github.com/sh05}
\date{\today}
% \subtitle{副題}
% \author{著者 \and 著者2}
% \institute[所属略称]{所属}

% テーマ
\usetheme{Madrid}

% その他テーマ
% \usetheme{Cuerna}
% \usetheme{PaloAlto}
% \usetheme{Rochester}
% \usetheme{split}
% \usetheme[nofirafonts]{focus}
% \definecolor{main}{rgb}{0.16, 0.32, 0.75} % focusの設定
% \definecolor{background}{RGB}{240, 247, 255} % focusの設定

% デフォルトが明朝なのでゴシック体にする
\renewcommand{\kanjifamilydefault}{\gtdefault}
% beamerではindentがうまく動作しないので上書き
\renewcommand{\indent}{\hspace*{2em}}
% 図表も日本語に
% caption に番号追加
\setbeamertemplate{caption}[numbered]
% caption 日本語
\captionsetup[figure]{labelfont={bf},name={図},labelsep=period}
\captionsetup[table]{labelfont={bf},name={表},labelsep=period}
% \renewcommand{\figurename}{図}
% \renewcommand{\tablename}{表}

% パッケージ読み込み
\usepackage{pxjahyper, multicol, multirow, tikz, listings}
% 描画パッケージtikzのライブラリ読み込み
\usetikzlibrary{positioning, angles, intersections, quotes}

\lstset{
  basicstyle=\ttfamily,
  showstringspaces=false,
  commentstyle=\color{red},
  keywordstyle=\color{blue}
}

% コードを埋め込む
\lstset{language=[LaTeX]tex}
% tex内にtexを埋め込む
\newsavebox{\codebox}% For storing listings

% 目次(Tabole of Contents)を表示する際の階層
\setcounter{tocdepth}{2}

% フッターをページ番号だけにする
% \setbeamertemplate{footline}[frame number]
% フッターのフォントサイズと太字
% \setbeamerfont{footline}{size=\Large,series=\bfseries}
% フッターのアイコンをなくす
\setbeamertemplate{navigation symbols}{}
% 目次のナンバリングを文字にする
\setbeamertemplate{section in toc}[sections numbered]
% スライドタイトルのフォントサイズ
\setbeamerfont{frametitle}{size=\huge}

% 再利用できるコマンドを設定
\newcommand{\beameris}{
    Beamer(ビーマー)は\LaTeX に基づき、プレゼンテーションを作成するためのクラスである。名称の Beamer はビデオプロジェクターを意味するドイツ語に由来している。 \cite{BeamerWi26:online}
}
\newcommand{\highlight}[2][cyan]{\tikz[baseline=(x.base)]{\node[rectangle,rounded corners,fill=#1!10](x){#2};}}
\newcommand{\assert}[2][blue]{
  \begin{figure}[!h]
    \centering
    \begin{tikzpicture}
      \node[rectangle, fill=#1!70, text=white, text width=10cm, text centered, rounded corners, inner sep=8pt]{#2};
    \end{tikzpicture}
  \end{figure}
}
% 各sectionの初めに自動で挿入するスライド
\AtBeginSection{
  \setbeamercolor{background canvas}{bg=beamer@blendedblue}
  \color{white}
  \frame[plain]{
      \indent \Huge{\insertsection}
  }
  \color{black}
  \setbeamercolor{background canvas}{bg=white}
}

% 各subsectionの初めに自動で挿入するスライド
\AtBeginSubsection{
  \frame{
    \frametitle{目次}
    \begin{multicols}{2}
      \tableofcontents[sections={1}, sectionstyle=show/shaded,subsectionstyle=show/shaded]
      \bigskip
      \tableofcontents[sections={2}, sectionstyle=show/shaded,subsectionstyle=show/shaded]
      \columnbreak
      \tableofcontents[sections={3}, sectionstyle=show/shaded,subsectionstyle=show/shaded]
      \bigskip
      \tableofcontents[sections={4}, sectionstyle=show/shaded,subsectionstyle=show/shaded]
    \end{multicols}
  } 
}

% コマンド定義
\include{commands}

% document内が文書本体
\begin{document}
\begin{frame}{}
  \frametitle{修士懇親会}
      aaaa
      aaaa
      aaaa
      aaaa
  \begin{itemize}
    \item 流れ
      \begin{enumerate}
          \pause
        \item 開始19:20
          \pause
        \item 乾杯
          \pause
        \item 全体で自己紹介
          \pause
        \item 人数が多ければbreakoutセッション
          \pause
        \item 終了未定、流れ解散
      \end{enumerate}
    \item 飲み物や食べ物は各自用意
    \item 中本は熱があるのでカメラオフで失礼します(早退するかも)
  \end{itemize}
\end{frame}

\begin{frame}{}
  \frametitle{自己紹介}
  \begin{multicols}{2}
    必ず含めてください
    \begin{enumerate}
      \item 名前
      \item 学年
      \item 研究室
    \end{enumerate}
    \columnbreak
    良ければ言ってください
    \begin{enumerate}
      \item 呼んで欲しい名前
      \item 自粛中に始めたこと
      \item 中間発表の出来
      \item 研究の進み具合
        \begin{itemize}
          \item 良い・普通・悪い
        \end{itemize}
    \end{enumerate}
  \end{multicols}
\end{frame}
% % プリアンブルで定義したタイトルなどをもとにタイトルスライドを自動生成
% \maketitle

% \begin{frame}{}
%   \frametitle{Outline}
%   この目次は自動生成されます。
%   \par
%   各セクション、サブセクションをクリックすると移動します。
%   \tableofcontents
% \end{frame}

% \begin{frame}{}
%   \frametitle{Outline}
%   段組みも可能です(multicols)
%     \begin{multicols}{2}
%       \tableofcontents[sections={1-2}]
%       \columnbreak
%       \tableofcontents[sections={3-4}]
%     \end{multicols}
% \end{frame}

% \section{イントロ1}
% \subsection{サブセクション1}
% \begin{frame}
%   \frametitle{\insertsection}
%   \insertsection
% \end{frame}
% \begin{frame}
%   \frametitle{\insertsubsection}
%   \insertsubsection
% \end{frame}
% \section{イントロ2}
% \subsection{サブセクション2}
% \begin{frame}
%   \frametitle{\insertsection}
%   \insertsection
% \end{frame}
% \begin{frame}
%   \frametitle{\insertsubsection}
%   \insertsubsection
% \end{frame}
% \section{イントロ3}
% \subsection{サブセクション3}
% \begin{frame}
%   \frametitle{\insertsection}
%   \insertsection
% \end{frame}
% \begin{frame}
%   \frametitle{\insertsubsection}
%   \insertsubsection
% \end{frame}
% \section{イントロ4}
% \subsection{サブセクション4}
% \begin{frame}
%   \frametitle{\insertsection}
%   \insertsection
% \end{frame}
% \begin{frame}
%   \frametitle{\insertsubsection}
%   \insertsubsection
% \end{frame}

% % \begin{frame}
% %   \frametitle{\insertsection}
% %   \assert{ We propose \textbf {Emotional Document Vector(EDV)}.  }
% %   \begin{itemize}
% %     \item EDV is KANSEI vector extracted from document vector.
% %     We use this to improve recommendation across multiple domains.
% %       We recommend works across seven domains, \highlight{\domains}.
% %     \item In experiment, we compare three methods, EDV and others and examine the need to composite vectors in each method.
% %     As a result, we found that EDV is highly effective and needs attention.
% %   \end{itemize}
% % \end{frame}


% % \begin{frame}
% %   \frametitle{\insertsubsection}
% %   \begin{figure}[!h]
% %     \centering
% %     \begin{tikzpicture}
% %       \tikzset{block/.style={rectangle, fill=blue!70, text=white, text width=10cm, text centered, rounded corners, inner sep=8pt}};
% %       \node[block] (a) {
% %         We propose \textbf {Emotional Document Vector(EDV)}.
% %       };
% %     \end{tikzpicture}
% %   \end{figure}
% %   \begin{itemize}
% %     \item EDV is KANSEI vector extracted from document vector.
% %     We use this to improve recommendation across multiple domains.
% %       We recommend works across seven domains, \highlight{\domains}.
% %     \item In experiment, we compare three methods, EDV and others and examine the need to composite vectors in each method.
% %     As a result, we found that EDV is highly effective and needs attention.
% %   \end{itemize}
% % \end{frame}


% % \section{texとして}
% % \include{overview}
% % \section{texとスライドとして}
% % \include{introduction}
% % \section{Materials and Methods}
% % \include{concept}
% % \include{matemeth}
% % \section{Experiment}
% % \include{experiment}
% % \section{Conculsion}
% % \include{conclusion}
% % \begin{frame}
    \frametitle{Refference}
    \scriptsize
    \beamertemplatetextbibitems
    \bibliographystyle{junsrt}
    \bibliography{ref}
\end{frame}

\end{document}
