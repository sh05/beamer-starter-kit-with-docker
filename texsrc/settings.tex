% タイトル
\title{Beamer Starter Kit with Docker}
\author{https://github.com/sh05}
\date{\today}
% \subtitle{副題}
% \author{著者 \and 著者2}
% \institute[所属略称]{所属}

% テーマ
\usetheme{Madrid}

% その他テーマ
% \usetheme{Cuerna}
% \usetheme{PaloAlto}
% \usetheme{Rochester}
% \usetheme{split}
% \usetheme[nofirafonts]{focus}
% \definecolor{main}{rgb}{0.16, 0.32, 0.75} % focusの設定
% \definecolor{background}{RGB}{240, 247, 255} % focusの設定

% デフォルトが明朝なのでゴシック体にする
\renewcommand{\kanjifamilydefault}{\gtdefault}
% beamerではindentがうまく動作しないので上書き
\renewcommand{\indent}{\hspace*{2em}}
% 図表も日本語に
% caption に番号追加
\setbeamertemplate{caption}[numbered]
% caption 日本語
\captionsetup[figure]{labelfont={bf},name={図},labelsep=period}
\captionsetup[table]{labelfont={bf},name={表},labelsep=period}
% \renewcommand{\figurename}{図}
% \renewcommand{\tablename}{表}

% パッケージ読み込み
\usepackage{pxjahyper, multicol, multirow, tikz, listings}
% 描画パッケージtikzのライブラリ読み込み
\usetikzlibrary{positioning, angles, intersections, quotes}

\lstset{
  basicstyle=\ttfamily,
  showstringspaces=false,
  commentstyle=\color{red},
  keywordstyle=\color{blue}
}

% コードを埋め込む
\lstset{language=[LaTeX]tex}
% tex内にtexを埋め込む
\newsavebox{\codebox}% For storing listings

% 目次(Tabole of Contents)を表示する際の階層
\setcounter{tocdepth}{2}

% フッターをページ番号だけにする
% \setbeamertemplate{footline}[frame number]
% フッターのフォントサイズと太字
% \setbeamerfont{footline}{size=\Large,series=\bfseries}
% フッターのアイコンをなくす
\setbeamertemplate{navigation symbols}{}
% 目次のナンバリングを文字にする
\setbeamertemplate{section in toc}[sections numbered]
% スライドタイトルのフォントサイズ
\setbeamerfont{frametitle}{size=\huge}

% 再利用できるコマンドを設定
\newcommand{\beameris}{
    Beamer(ビーマー)は\LaTeX に基づき、プレゼンテーションを作成するためのクラスである。名称の Beamer はビデオプロジェクターを意味するドイツ語に由来している。 \cite{BeamerWi26:online}
}
\newcommand{\highlight}[2][cyan]{\tikz[baseline=(x.base)]{\node[rectangle,rounded corners,fill=#1!10](x){#2};}}
\newcommand{\assert}[2][blue]{
  \begin{figure}[!h]
    \centering
    \begin{tikzpicture}
      \node[rectangle, fill=#1!70, text=white, text width=10cm, text centered, rounded corners, inner sep=8pt]{#2};
    \end{tikzpicture}
  \end{figure}
}
% 各sectionの初めに自動で挿入するスライド
\AtBeginSection{
  \setbeamercolor{background canvas}{bg=beamer@blendedblue}
  \color{white}
  \frame[plain]{
      \indent \Huge{\insertsection}
  }
  \color{black}
  \setbeamercolor{background canvas}{bg=white}
}

% 各subsectionの初めに自動で挿入するスライド
\AtBeginSubsection{
  \frame{
    \frametitle{目次}
    \begin{multicols}{2}
      \tableofcontents[sections={1}, sectionstyle=show/shaded,subsectionstyle=show/shaded]
      \bigskip
      \tableofcontents[sections={2}, sectionstyle=show/shaded,subsectionstyle=show/shaded]
      \columnbreak
      \tableofcontents[sections={3}, sectionstyle=show/shaded,subsectionstyle=show/shaded]
      \bigskip
      \tableofcontents[sections={4}, sectionstyle=show/shaded,subsectionstyle=show/shaded]
    \end{multicols}
  } 
}
